\documentclass[letterpaper,10pt,titlepage]{article}

\usepackage{graphicx}
\usepackage{amssymb}
\usepackage{amsmath}
\usepackage{amsthm}

\usepackage{alltt}
\usepackage{float}
\usepackage{color}
\usepackage{url}
\usepackage{listings}

\usepackage{balance}
\usepackage[TABBOTCAP, tight]{subfigure}
\usepackage{enumitem}
\usepackage{pstricks, pst-node}

\usepackage{geometry}
\geometry{textheight=8.5in, textwidth=6in}

%random comment

\newcommand{\cred}[1]{{\color{red}#1}}
\newcommand{\cblue}[1]{{\color{blue}#1}}

\usepackage{hyperref}
\usepackage{geometry}

\def\name{Brandon Lee}

%pull in the necessary preamble matter for pygments output
% \usepackage{fancyvrb}
\usepackage{color}
\usepackage[latin1]{inputenc}


\makeatletter
\def\PY@reset{\let\PY@it=\relax \let\PY@bf=\relax%
    \let\PY@ul=\relax \let\PY@tc=\relax%
    \let\PY@bc=\relax \let\PY@ff=\relax}
\def\PY@tok#1{\csname PY@tok@#1\endcsname}
\def\PY@toks#1+{\ifx\relax#1\empty\else%
    \PY@tok{#1}\expandafter\PY@toks\fi}
\def\PY@do#1{\PY@bc{\PY@tc{\PY@ul{%
    \PY@it{\PY@bf{\PY@ff{#1}}}}}}}
\def\PY#1#2{\PY@reset\PY@toks#1+\relax+\PY@do{#2}}

\expandafter\def\csname PY@tok@gd\endcsname{\def\PY@tc##1{\textcolor[rgb]{0.63,0.00,0.00}{##1}}}
\expandafter\def\csname PY@tok@gu\endcsname{\let\PY@bf=\textbf\def\PY@tc##1{\textcolor[rgb]{0.50,0.00,0.50}{##1}}}
\expandafter\def\csname PY@tok@gt\endcsname{\def\PY@tc##1{\textcolor[rgb]{0.00,0.25,0.82}{##1}}}
\expandafter\def\csname PY@tok@gs\endcsname{\let\PY@bf=\textbf}
\expandafter\def\csname PY@tok@gr\endcsname{\def\PY@tc##1{\textcolor[rgb]{1.00,0.00,0.00}{##1}}}
\expandafter\def\csname PY@tok@cm\endcsname{\let\PY@it=\textit\def\PY@tc##1{\textcolor[rgb]{0.25,0.50,0.50}{##1}}}
\expandafter\def\csname PY@tok@vg\endcsname{\def\PY@tc##1{\textcolor[rgb]{0.10,0.09,0.49}{##1}}}
\expandafter\def\csname PY@tok@m\endcsname{\def\PY@tc##1{\textcolor[rgb]{0.40,0.40,0.40}{##1}}}
\expandafter\def\csname PY@tok@mh\endcsname{\def\PY@tc##1{\textcolor[rgb]{0.40,0.40,0.40}{##1}}}
\expandafter\def\csname PY@tok@go\endcsname{\def\PY@tc##1{\textcolor[rgb]{0.50,0.50,0.50}{##1}}}
\expandafter\def\csname PY@tok@ge\endcsname{\let\PY@it=\textit}
\expandafter\def\csname PY@tok@vc\endcsname{\def\PY@tc##1{\textcolor[rgb]{0.10,0.09,0.49}{##1}}}
\expandafter\def\csname PY@tok@il\endcsname{\def\PY@tc##1{\textcolor[rgb]{0.40,0.40,0.40}{##1}}}
\expandafter\def\csname PY@tok@cs\endcsname{\let\PY@it=\textit\def\PY@tc##1{\textcolor[rgb]{0.25,0.50,0.50}{##1}}}
\expandafter\def\csname PY@tok@cp\endcsname{\def\PY@tc##1{\textcolor[rgb]{0.74,0.48,0.00}{##1}}}
\expandafter\def\csname PY@tok@gi\endcsname{\def\PY@tc##1{\textcolor[rgb]{0.00,0.63,0.00}{##1}}}
\expandafter\def\csname PY@tok@gh\endcsname{\let\PY@bf=\textbf\def\PY@tc##1{\textcolor[rgb]{0.00,0.00,0.50}{##1}}}
\expandafter\def\csname PY@tok@ni\endcsname{\let\PY@bf=\textbf\def\PY@tc##1{\textcolor[rgb]{0.60,0.60,0.60}{##1}}}
\expandafter\def\csname PY@tok@nl\endcsname{\def\PY@tc##1{\textcolor[rgb]{0.63,0.63,0.00}{##1}}}
\expandafter\def\csname PY@tok@nn\endcsname{\let\PY@bf=\textbf\def\PY@tc##1{\textcolor[rgb]{0.00,0.00,1.00}{##1}}}
\expandafter\def\csname PY@tok@no\endcsname{\def\PY@tc##1{\textcolor[rgb]{0.53,0.00,0.00}{##1}}}
\expandafter\def\csname PY@tok@na\endcsname{\def\PY@tc##1{\textcolor[rgb]{0.49,0.56,0.16}{##1}}}
\expandafter\def\csname PY@tok@nb\endcsname{\def\PY@tc##1{\textcolor[rgb]{0.00,0.50,0.00}{##1}}}
\expandafter\def\csname PY@tok@nc\endcsname{\let\PY@bf=\textbf\def\PY@tc##1{\textcolor[rgb]{0.00,0.00,1.00}{##1}}}
\expandafter\def\csname PY@tok@nd\endcsname{\def\PY@tc##1{\textcolor[rgb]{0.67,0.13,1.00}{##1}}}
\expandafter\def\csname PY@tok@ne\endcsname{\let\PY@bf=\textbf\def\PY@tc##1{\textcolor[rgb]{0.82,0.25,0.23}{##1}}}
\expandafter\def\csname PY@tok@nf\endcsname{\def\PY@tc##1{\textcolor[rgb]{0.00,0.00,1.00}{##1}}}
\expandafter\def\csname PY@tok@si\endcsname{\let\PY@bf=\textbf\def\PY@tc##1{\textcolor[rgb]{0.73,0.40,0.53}{##1}}}
\expandafter\def\csname PY@tok@s2\endcsname{\def\PY@tc##1{\textcolor[rgb]{0.73,0.13,0.13}{##1}}}
\expandafter\def\csname PY@tok@vi\endcsname{\def\PY@tc##1{\textcolor[rgb]{0.10,0.09,0.49}{##1}}}
\expandafter\def\csname PY@tok@nt\endcsname{\let\PY@bf=\textbf\def\PY@tc##1{\textcolor[rgb]{0.00,0.50,0.00}{##1}}}
\expandafter\def\csname PY@tok@nv\endcsname{\def\PY@tc##1{\textcolor[rgb]{0.10,0.09,0.49}{##1}}}
\expandafter\def\csname PY@tok@s1\endcsname{\def\PY@tc##1{\textcolor[rgb]{0.73,0.13,0.13}{##1}}}
\expandafter\def\csname PY@tok@sh\endcsname{\def\PY@tc##1{\textcolor[rgb]{0.73,0.13,0.13}{##1}}}
\expandafter\def\csname PY@tok@sc\endcsname{\def\PY@tc##1{\textcolor[rgb]{0.73,0.13,0.13}{##1}}}
\expandafter\def\csname PY@tok@sx\endcsname{\def\PY@tc##1{\textcolor[rgb]{0.00,0.50,0.00}{##1}}}
\expandafter\def\csname PY@tok@bp\endcsname{\def\PY@tc##1{\textcolor[rgb]{0.00,0.50,0.00}{##1}}}
\expandafter\def\csname PY@tok@c1\endcsname{\let\PY@it=\textit\def\PY@tc##1{\textcolor[rgb]{0.25,0.50,0.50}{##1}}}
\expandafter\def\csname PY@tok@kc\endcsname{\let\PY@bf=\textbf\def\PY@tc##1{\textcolor[rgb]{0.00,0.50,0.00}{##1}}}
\expandafter\def\csname PY@tok@c\endcsname{\let\PY@it=\textit\def\PY@tc##1{\textcolor[rgb]{0.25,0.50,0.50}{##1}}}
\expandafter\def\csname PY@tok@mf\endcsname{\def\PY@tc##1{\textcolor[rgb]{0.40,0.40,0.40}{##1}}}
\expandafter\def\csname PY@tok@err\endcsname{\def\PY@bc##1{\setlength{\fboxsep}{0pt}\fcolorbox[rgb]{1.00,0.00,0.00}{1,1,1}{\strut ##1}}}
\expandafter\def\csname PY@tok@kd\endcsname{\let\PY@bf=\textbf\def\PY@tc##1{\textcolor[rgb]{0.00,0.50,0.00}{##1}}}
\expandafter\def\csname PY@tok@ss\endcsname{\def\PY@tc##1{\textcolor[rgb]{0.10,0.09,0.49}{##1}}}
\expandafter\def\csname PY@tok@sr\endcsname{\def\PY@tc##1{\textcolor[rgb]{0.73,0.40,0.53}{##1}}}
\expandafter\def\csname PY@tok@mo\endcsname{\def\PY@tc##1{\textcolor[rgb]{0.40,0.40,0.40}{##1}}}
\expandafter\def\csname PY@tok@kn\endcsname{\let\PY@bf=\textbf\def\PY@tc##1{\textcolor[rgb]{0.00,0.50,0.00}{##1}}}
\expandafter\def\csname PY@tok@mi\endcsname{\def\PY@tc##1{\textcolor[rgb]{0.40,0.40,0.40}{##1}}}
\expandafter\def\csname PY@tok@gp\endcsname{\let\PY@bf=\textbf\def\PY@tc##1{\textcolor[rgb]{0.00,0.00,0.50}{##1}}}
\expandafter\def\csname PY@tok@o\endcsname{\def\PY@tc##1{\textcolor[rgb]{0.40,0.40,0.40}{##1}}}
\expandafter\def\csname PY@tok@kr\endcsname{\let\PY@bf=\textbf\def\PY@tc##1{\textcolor[rgb]{0.00,0.50,0.00}{##1}}}
\expandafter\def\csname PY@tok@s\endcsname{\def\PY@tc##1{\textcolor[rgb]{0.73,0.13,0.13}{##1}}}
\expandafter\def\csname PY@tok@kp\endcsname{\def\PY@tc##1{\textcolor[rgb]{0.00,0.50,0.00}{##1}}}
\expandafter\def\csname PY@tok@w\endcsname{\def\PY@tc##1{\textcolor[rgb]{0.73,0.73,0.73}{##1}}}
\expandafter\def\csname PY@tok@kt\endcsname{\def\PY@tc##1{\textcolor[rgb]{0.69,0.00,0.25}{##1}}}
\expandafter\def\csname PY@tok@ow\endcsname{\let\PY@bf=\textbf\def\PY@tc##1{\textcolor[rgb]{0.67,0.13,1.00}{##1}}}
\expandafter\def\csname PY@tok@sb\endcsname{\def\PY@tc##1{\textcolor[rgb]{0.73,0.13,0.13}{##1}}}
\expandafter\def\csname PY@tok@k\endcsname{\let\PY@bf=\textbf\def\PY@tc##1{\textcolor[rgb]{0.00,0.50,0.00}{##1}}}
\expandafter\def\csname PY@tok@se\endcsname{\let\PY@bf=\textbf\def\PY@tc##1{\textcolor[rgb]{0.73,0.40,0.13}{##1}}}
\expandafter\def\csname PY@tok@sd\endcsname{\let\PY@it=\textit\def\PY@tc##1{\textcolor[rgb]{0.73,0.13,0.13}{##1}}}

\def\PYZbs{\char`\\}
\def\PYZus{\char`\_}
\def\PYZob{\char`\{}
\def\PYZcb{\char`\}}
\def\PYZca{\char`\^}
\def\PYZam{\char`\&}
\def\PYZlt{\char`\<}
\def\PYZgt{\char`\>}
\def\PYZsh{\char`\#}
\def\PYZpc{\char`\%}
\def\PYZdl{\char`\$}
\def\PYZti{\char`\~}
% for compatibility with earlier versions
\def\PYZat{@}
\def\PYZlb{[}
\def\PYZrb{]}
\makeatother


%% The following metadata will show up in the PDF properties
\hypersetup{
  colorlinks = true,
  urlcolor = black,
  pdfauthor = {\name},
  pdfkeywords = {CS444 ``operating systems'' files filesystem I/O},
  pdftitle = {CS 444 Project 1},
  pdfsubject = {CS 444 Project 1},
  pdfpagemode = UseNone
}

\begin{document}

\begin{titlepage}
    \begin{center}
        \vspace*{3.5cm}

        \textbf{Project 1}

        \vspace{0.5cm}

        \textbf{Brandon Lee}

        \vspace{0.8cm}

        CS 444\\
        Spring 2016\\
        10 April 2016\\

        \vspace{1cm}

        \textbf{Abstract}\\

        \vspace{0.5cm}

        Linux processes and treads are essential components in developing a fundamental understanding for low level systems.  Concurrent threads with multi-process operations and inter-process communication are critical in writing safe code.  The scope of the Consumer-Producer Problem lies a multi-process synchronization problem where safety precautions need to be made in order to avoid resource issues.  The following approach utilizes pthread mutex locks and process waiting with various conditions to ensure valid resource access.  The results include a constantly running buffer with producer and consumer functions adding and taking from a shared buffer resource.

        \vfill



    \end{center}
\end{titlepage}

\newpage

\section{Log of Commands}

In order to set up the QEMU and Yocto, I had to first create a repo for all my personal coursework.  I created a private repo on GitHub and ran:

\begin{lstlisting}
$ git clone https://github.com/brandonlee503/Operating-Systems-II.git
\end{lstlisting}


I ran this to create a directory at /scratch/spring2016/cs444-074\\

Next I cloned the linux project into this newly created directory with:

\begin{lstlisting}
$ git clone git://git.yoctoproject.org/linux-yocto-3.14
\end{lstlisting}

Also checked out the correct branch with:

\begin{lstlisting}
$ git checkout v3.14.26
\end{lstlisting}


Next I sourced the environment configuration script with the command:

\begin{lstlisting}
$ source /scratch/opt/environment-setup-i586-poky-linux
\end{lstlisting}

Now for running qemu in debug mode, in the respective directory, I ran the long command:

\begin{lstlisting}
$ qemu-system-i386 -gdb tcp::5574 -S -nographic -kernel bzImage-qemux86.bin
-drive file=core-image-lsb-sdk-qemux86.ext3,if=virtio -enable-kvm -net none -usb
-localtime --no-reboot --append "root=/dev/vda rw console=ttyS0 debug"
\end{lstlisting}

And now with qemu running in debug mode, to start up the VM, I opened a new terminal tab and ran the commands to source and target the VM:
\begin{lstlisting}
$ source /scratch/opt/environment-setup-i586-poky-linux
$ $GDB
$ target remote :5574
$ continue
\end{lstlisting}

Now back in the initial window, QEMU/YOCTO should be building. But before we can use Yocto, we need to build it first.  To do so I ran the following commands to copy the correct .config file and build it with 4 threads.

\begin{lstlisting}
$ cp /scratch/spring2015/files/config-3.14.26-yocto-qemu
/scratch/spring2016/cs444-074/linux-yocto-3.14
$ make -j4 all
\end{lstlisting}

And to finally make sure Yocto correctly boots up in Qemu, I ran the above commands to run qemu in debug mode then launch, logged in to root and ran the following command to get the following response.

\begin{lstlisting}
$ uname -r
$ 3.14.26ltsi-yocto-standard
\end{lstlisting}

In the end, I wrote a quick script to get most of the setup done in one command, however still need to remote target from GDB from another terminal window:

\begin{lstlisting}
#!/bin/bash
$ echo "Setting up..."
$ source /scratch/opt/environment-setup-i586-poky-linux
$ cd /scratch/spring2016/cs444-074/linux-yocto-3.14/

$ qemu-system-i386 -gdb tcp::5574 -S -nographic -kernel bzImage-qemux86.bin -drive
file=core-image-lsb-sdk-qemux86.ext3,if=virtio
-enable-kvm -net none -usb
-localtime --no-reboot --append "root=/dev/vda rw console=ttyS0 debug"
\end{lstlisting}

\section{Concurrency Solution Writeup}

My concurrency solution utilizes pthead mutex locks and waiting conditions in order to synchronously access the buffer resource through both the producer and consumer threads.  Essentially the program starts off with a producer thread creating new values and adding them to the buffer so long as there exists space.  When there is no longer any space, the producer waits for the consumer thread to start taking from the buffer.  As the consumer takes a value from the buffer, it increments to the next index in the buffer.  The consumer thread waits as the producer creates a new value, filling the index in which the consumer just took a value from.  This method ensures that the producer never adds to a full buffer and the consumer always has a value to consume from the buffer.


\section{QEMU Command-Line Flags}

\begin{itemize}
  \item -gdb: Wait for gdb connection on device tcp::5574.
  \item -S: In order to use gdb, -s flag will wait for a GDB connection.
  \item -nographic: Disable graphical output.
  \item -kernel: Provide Linux kernel image without having to make a full bootable image.
  \item -drive: Define a new drive file, the “file=” option defines which disk image to use with the drive.
  \item -enable-kvm: Enable full KVM virtualization support.
  \item -net: Set any necessary network options.
  \item -usb: Enable the USB driver.
  \item -localtime: Enable system to be set at local time.
  \item -no-reboot: Exit instead of rebooting.
  \item -append: Give kernel command line arguments/ Use appended command line as kernel command line.

\end{itemize}

\newpage

\section{Concurrency Questions}

\textit{What do you think the main point of this assignment is?}\\

I thought that the main point of this first concurrency
assignment was to develop a deeper understanding of
multi-process operations and the necessary precautions of
synchronization in order for safe inter-process communication.
I also think that the purpose of this assignment was to dive into pthreads and a refresher on signal communication.\\

\textit{How did you personally approach the problem? Design decisions, algorithm, etc.}\\

At first I was pretty nervous on approaching the problem since I had little understanding of exactly what we were supposed to do.  However after slowing down and thoroughly reading the instructions, I got a better general idea on what to do.  I started by writing some skeleton code for all the possible necessary functions I would need in order to meet the assignment requirements and went from there.  I created two structs (bufferData and bufferArray) to represent the buffer the consumer/producer would be working with, a producer function, a consumer function, a randomizer function (with helper functions for Mersenne Twister and rdrand) and some other boilerplate functions.\\

My program is designed in such a way that when the main function initialized some threads and signals, I’d create two separate threads for the producer and consumer, the producer would only need to run once as when it runs, it keeps running and simply waiting when the buffer was full, I created the consumer function to only take one element at a time so I put the function into a while loop in main.  The consumer function would also wait for the producer function to create items before proceeding in order to keep the whole operation going.  The threads would wait for each other through various conditions and kept access modular through mutex locks.\\

\textit{How did you ensure your solution was correct? Testing details, for instance.}\\

I ensured that my solution was correct through printing multiple trace statements and values throughout the whole development process.  After much strife, I was able to come to a solution I was satisfied with.  To visually evaluate my solution, I print out the consumption value every time the consumer takes out an item.  I also display the entire buffer for each iteration of the whole process so the user can visualize the consumer going through the entire buffer, index by index.  Meanwhile the producer is right behind the consumer, generating a new value after a value is taken out.  This solution ensures that the consumer never draws from an empty buffer and the producer never adds to a full buffer.\\

\textit{What did you learn?}\\

From this assignment I learned much on pthreads and the various methods of synchronization.  I learned a bunch on incorporating assembly into C code as I have never done anything of this sort before.  Also at the time, I was not aware of the rdrand template given to us so I had to figure most of that out on my own.  All in all, I learned a bunch on Linux processes and pthreads.\\

\section{Version Control Log}

\begin{tabular}{lll} \textbf{Author}
     & \textbf{Date}
     & \textbf{Message}

\\ \hline
Brandon Lee & 2016-03-31 & Initial commit \\ \hline
Brandon Lee & 2016-04-04 & Add gitignore and get initial kernel booting on VM \\ \hline
Brandon Lee & 2016-04-04 & Add skeleton layout for concurrency assignment 1 \\ \hline
Brandon Lee & 2016-04-04 & Include Mersenne Twister and incorporate into skeleton \\ \hline
Brandon Lee & 2016-04-04 & Add gitignore file \\ \hline
Brandon Lee & 2016-04-04 & Comment out main() \\ \hline
Brandon Lee & 2016-04-04 & Add makefile \\ \hline
Brandon Lee & 2016-04-08 & Modify makefile flags \\ \hline
Brandon Lee & 2016-04-08 & Implement Mersenne Twister and ASM instruction rdrand (only partially working\ldots{}) \\ \hline
Brandon Lee & 2016-04-08 & Fix randomNumberGenerator() \\ \hline
Brandon Lee & 2016-04-08 & Update bufferArray struct and add pthread condition globals \\ \hline
Brandon Lee & 2016-04-08 & Add signal catching \\ \hline
Brandon Lee & 2016-04-08 & Initialize pthreads and signals \\ \hline
Brandon Lee & 2016-04-08 & First attempt to create pthreads \\ \hline
Brandon Lee & 2016-04-08 & Update buffer name and add comments \\ \hline
Brandon Lee & 2016-04-08 & First implementation attempt - produce() \\ \hline
Brandon Lee & 2016-04-08 & Add comment reference \\ \hline
Brandon Lee & 2016-04-09 & Stupid bug - Add trace statements and comments \\ \hline
Brandon Lee & 2016-04-09 & FINALLY FOUND BUG \\ \hline
Brandon Lee & 2016-04-09 & First implementation attempt - consume() \\ \hline
Brandon Lee & 2016-04-09 & Add function pointers for pthread\_create() \\ \hline
Brandon Lee & 2016-04-09 & Cleanup trace statements/comments and print buffer for testing \\ \hline
Brandon Lee & 2016-04-09 & Add bufferHasSpace() \\ \hline
Brandon Lee & 2016-04-09 & Experiment with produce() \\ \hline
Brandon Lee & 2016-04-09 & Experiment with consume \\ \hline
Brandon Lee & 2016-04-09 & Should be functional now.. Clean experimental code and polish comments \\ \hline
Brandon Lee & 2016-04-10 & Add Weekly Summary 1 and 2 \\ \hline
\end{tabular}

\section{Work Log}

I started work on about exactly a week ago when I was attempting to get the VM to boot.  Through following the instructions specified on the class page, I was able to get everything functionally working and pushed it all in one commit.  Afterwards I began work on the concurrency part of the assignment.  First of all, I created a skeleton to figure out what core structs and functions I would need.  I started out with working on incorporating Mersenne Twister and rdrnd into the random function.  Because I was unaware that there was some rdrand sample code on our class page, I was stuck for a while attempting to figure it out on my own for the longest time.  After getting the random function working, I began work on setting up signals and pthreads.  Finally after that I began work on connecting everything into the producer and consumer functions that would ultimately be the main bulk of the challenge within this assignment.  After some heavy debugging, I was able to get what I perceive as a valid solution.

\end{document}
