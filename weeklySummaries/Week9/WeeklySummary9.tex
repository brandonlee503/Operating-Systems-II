\documentclass[letterpaper,10pt,titlepage]{article}

\usepackage{graphicx}
\usepackage{amssymb}
\usepackage{amsmath}
\usepackage{amsthm}

\usepackage{alltt}
\usepackage{float}
\usepackage{color}
\usepackage{url}

\usepackage{balance}
\usepackage[TABBOTCAP, tight]{subfigure}
\usepackage{enumitem}
\usepackage{pstricks, pst-node}

\usepackage{geometry}
\geometry{textheight=8.5in, textwidth=6in}

\newcommand{\cred}[1]{{\color{red}#1}}
\newcommand{\cblue}[1]{{\color{blue}#1}}

\usepackage{hyperref}
\usepackage{geometry}

\def\name{Brandon Lee}

%% The following metadata will show up in the PDF properties
\hypersetup{
  colorlinks = true,
  urlcolor = black,
  pdfauthor = {\name},
  pdfkeywords = {cs444 ``operating systems'' files filesystem I/O},
  pdftitle = {CS 444 Weekly Summary 9},
  pdfsubject = {CS 444 Weekly Summary 9},
  pdfpagemode = UseNone
}

\begin{document}

\begin{titlepage}
    \begin{center}
        \vspace*{3.5cm}

        \textbf{Weekly Summary 9}

        \vspace{0.5cm}

        \textbf{Brandon Lee}

        \vspace{0.8cm}

        CS 444\\
        Spring 2016\\
        29 May 2016\\

        \vspace{1cm}

        \textbf{Abstract}\\

        \vspace{0.5cm}

        The objective of this document is to illustrate the comprehension of chapters eleven, sixteen, ten, and thirteen in Robert Love's “Linux Kernel Development” (2010) through the form of Rhetorical Precis.

        \vfill

    \end{center}
\end{titlepage}

\newpage

\section{Rhetorical Precis}

Robert Love's eleventh, sixteenth, tenth, and thirteenth chapters of "Linux Kernel Development" (2010) explains the fundamentals of time management, page caching, kernel synchronization, and virtual file systems.  Love supports his claims through various descriptions of these aspects and the details of these foundations, specifically with timers, caches, semophores\/mutexes, and VFS objects.  Love's purpose is to portray these core foundations in order to promote proper usage of such concepts and tools at disposal.  Love writes to an audience of computer savvy readers as the language he utilizes much background knowledge context of the Linux kernel.

\end{document}
