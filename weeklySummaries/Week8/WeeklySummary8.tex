\documentclass[letterpaper,10pt,titlepage]{article}

\usepackage{graphicx}
\usepackage{amssymb}
\usepackage{amsmath}
\usepackage{amsthm}

\usepackage{alltt}
\usepackage{float}
\usepackage{color}
\usepackage{url}

\usepackage{balance}
\usepackage[TABBOTCAP, tight]{subfigure}
\usepackage{enumitem}
\usepackage{pstricks, pst-node}

\usepackage{geometry}
\geometry{textheight=8.5in, textwidth=6in}

\newcommand{\cred}[1]{{\color{red}#1}}
\newcommand{\cblue}[1]{{\color{blue}#1}}

\usepackage{hyperref}
\usepackage{geometry}

\def\name{Brandon Lee}

%% The following metadata will show up in the PDF properties
\hypersetup{
  colorlinks = true,
  urlcolor = black,
  pdfauthor = {\name},
  pdfkeywords = {cs311 ``operating systems'' files filesystem I/O},
  pdftitle = {CS 444 Weekly Summary 8},
  pdfsubject = {CS 444 Weekly Summary 8},
  pdfpagemode = UseNone
}

\begin{document}

\begin{titlepage}
    \begin{center}
        \vspace*{3.5cm}

        \textbf{Weekly Summary 8}

        \vspace{0.5cm}

        \textbf{Brandon Lee}

        \vspace{0.8cm}

        CS 444\\
        Spring 2016\\
        22 May 2016\\

        \vspace{1cm}

        \textbf{Abstract}\\

        \vspace{0.5cm}

        The objective of this document is to illustrate the comprehension of chapters eleven and sixteen in Robert Love's “Linux Kernel Development” (2010) through the form of Rhetorical Precis.


        \vfill



    \end{center}
\end{titlepage}

\newpage

\section{Rhetorical Precis}

Robert Love's eleventh and sixteenth chapter of "Linux Kernel Development" (2010) explains the fundamentals of time management as well as the various software components of pages. Love supports his claims through various aspects of kernel timers such as tick rate and hardware clocks as well as the foundations of pages through page caching, buffer caching, kernel and flusher threads. Love's purpose is to portray these core kernel foundations in order to promote the proper usage of such concepts and tools at the user's disposal. Love writes to an audience of computer engineers as the context and language he utilizes is of particularly higher level kernel knowledge.

\end{document}
