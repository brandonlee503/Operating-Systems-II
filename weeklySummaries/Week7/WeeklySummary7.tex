\documentclass[letterpaper,10pt,titlepage]{article}

\usepackage{graphicx}
\usepackage{amssymb}
\usepackage{amsmath}
\usepackage{amsthm}

\usepackage{alltt}
\usepackage{float}
\usepackage{color}
\usepackage{url}

\usepackage{balance}
\usepackage[TABBOTCAP, tight]{subfigure}
\usepackage{enumitem}
\usepackage{pstricks, pst-node}

\usepackage{geometry}
\geometry{textheight=8.5in, textwidth=6in}

\newcommand{\cred}[1]{{\color{red}#1}}
\newcommand{\cblue}[1]{{\color{blue}#1}}

\usepackage{hyperref}
\usepackage{geometry}

\def\name{Brandon Lee}

%% The following metadata will show up in the PDF properties
\hypersetup{
  colorlinks = true,
  urlcolor = black,
  pdfauthor = {\name},
  pdfkeywords = {cs311 ``operating systems'' files filesystem I/O},
  pdftitle = {CS 444 Weekly Summary 7},
  pdfsubject = {CS 444 Weekly Summary 7},
  pdfpagemode = UseNone
}

\begin{document}

\begin{titlepage}
    \begin{center}
        \vspace*{3.5cm}

        \textbf{Weekly Summary 7}

        \vspace{0.5cm}

        \textbf{Brandon Lee}

        \vspace{0.8cm}

        CS 444\\
        Spring 2016\\
        8 May 2016\\

        \vspace{1cm}

        \textbf{Abstract}\\

        \vspace{0.5cm}

        The objective of this document is to illustrate the comprehension of Robert Love's “Linux Kernel Development” (2010) through the form of Rhetorical Precis.


        \vfill



    \end{center}
\end{titlepage}

\newpage

\section{Rhetorical Precis}

This week we did not have any readings to do. However because Kevin mentioned it in class, this is my submission.

\end{document}
