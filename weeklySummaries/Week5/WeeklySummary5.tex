\documentclass[letterpaper,10pt,titlepage]{article}

\usepackage{graphicx}
\usepackage{amssymb}
\usepackage{amsmath}
\usepackage{amsthm}

\usepackage{alltt}
\usepackage{float}
\usepackage{color}
\usepackage{url}

\usepackage{balance}
\usepackage[TABBOTCAP, tight]{subfigure}
\usepackage{enumitem}
\usepackage{pstricks, pst-node}

\usepackage{geometry}
\geometry{textheight=8.5in, textwidth=6in}

\newcommand{\cred}[1]{{\color{red}#1}}
\newcommand{\cblue}[1]{{\color{blue}#1}}

\usepackage{hyperref}
\usepackage{geometry}

\def\name{Brandon Lee}

%% The following metadata will show up in the PDF properties
\hypersetup{
  colorlinks = true,
  urlcolor = black,
  pdfauthor = {\name},
  pdfkeywords = {cs311 ``operating systems'' files filesystem I/O},
  pdftitle = {CS 444 Weekly Summary 5},
  pdfsubject = {CS 444 Weekly Summary 5},
  pdfpagemode = UseNone
}

\begin{document}

\begin{titlepage}
    \begin{center}
        \vspace*{3.5cm}

        \textbf{Weekly Summary 5}

        \vspace{0.5cm}

        \textbf{Brandon Lee}

        \vspace{0.8cm}

        CS 444\\
        Spring 2016\\
        1 May 2016\\

        \vspace{1cm}

        \textbf{Abstract}\\

        \vspace{0.5cm}

        The objective of this document is to illustrate the comprehension of chapters eight and twelve in Robert Love's “Linux Kernel Development” (2010) through the form of Rhetorical Precis.


        \vfill



    \end{center}
\end{titlepage}

\newpage

\section{Rhetorical Precis}

Robert Love's eighth and twelfth chapter of "Linux Kernel Development" (2010) explains the foundational concepts of interrupt bottom halves and memory management concepts of the modern operating system. Love supports his claims through explanations of various fundamental bottom half mechanisms such as softirqs and tasklets as well as memory management through numerous methods to obtain inside the kernel. Love's purpose is to illustrate these core components in the kernel in order to promote the proper usages of the various tools at user disposal. Love writes to an audience of knowledgeable readers, particularly computer programmers whom have had experience in low level architecture.

\end{document}
