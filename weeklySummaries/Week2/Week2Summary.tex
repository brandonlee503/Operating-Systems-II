\documentclass[letterpaper,10pt,titlepage]{article}

\usepackage{graphicx}
\usepackage{amssymb}
\usepackage{amsmath}
\usepackage{amsthm}

\usepackage{alltt}
\usepackage{float}
\usepackage{color}
\usepackage{url}

\usepackage{balance}
\usepackage[TABBOTCAP, tight]{subfigure}
\usepackage{enumitem}
\usepackage{pstricks, pst-node}

\usepackage{geometry}
\geometry{textheight=8.5in, textwidth=6in}

%random comment

\newcommand{\cred}[1]{{\color{red}#1}}
\newcommand{\cblue}[1]{{\color{blue}#1}}

\usepackage{hyperref}
\usepackage{geometry}

\def\name{Brandon Lee}

%pull in the necessary preamble matter for pygments output
% \input{pygments.tex}

%% The following metadata will show up in the PDF properties
\hypersetup{
  colorlinks = true,
  urlcolor = black,
  pdfauthor = {\name},
  pdfkeywords = {cs444 ``operating systems'' files filesystem I/O},
  pdftitle = {CS 444 Weekly Summary 2},
  pdfsubject = {CS 444 Weekly Summary 2},
  pdfpagemode = UseNone
}

\begin{document}

\begin{titlepage}
    \begin{center}
        \vspace*{3.5cm}

        \textbf{Weekly Summary 2}

        \vspace{0.5cm}

        \textbf{Brandon Lee}

        \vspace{0.8cm}

        CS 444\\
        Spring 2016\\
        10 April 2016\\

        \vspace{1cm}

        \textbf{Abstract}\\

        \vspace{0.5cm}

        The objective of this document is to illustrate comprehension of chapters three and four of Robert Love's "Linux Kernel Development" (2010) through the succinct form of Rhetorical Precis.

        \vfill



    \end{center}
\end{titlepage}

\newpage

\section{Rhetorical Precis}

Robert Love's third and fourth chapter of "Linux Kernel Development" (2010) explains the crucial concepts of Linux processes and process management as well as how essential they are for multitasking operating systems.  Love supports his claims through various explanations of process fundamentals such as process descriptors, process state, and the lifecycle of a process as well as process management concepts including process scheduling, policies, and implementation examples.  Love's purpose is to illustrate the core basics of processes and process scheduling and how essentially intertwined they are to the entirety of the operating system.  Through the various code implementations throughout the two chapters, it is obvious that Love's intended audience are computer science enthusiasts whom are fluent in C and programming syntax.


\end{document}
