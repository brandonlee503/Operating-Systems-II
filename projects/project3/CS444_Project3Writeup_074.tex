\documentclass[letterpaper,10pt,titlepage]{article}

\setlength{\parindent}{0px}

\usepackage{graphicx}
\usepackage{amssymb}
\usepackage{amsmath}
\usepackage{amsthm}

\usepackage{alltt}
\usepackage{float}
\usepackage{color}
\usepackage{url}
\usepackage{listings}

\usepackage{balance}
\usepackage[TABBOTCAP, tight]{subfigure}
\usepackage{enumitem}
\usepackage{pstricks, pst-node}

\usepackage{geometry}
\geometry{textheight=8.5in, textwidth=6in}

\newcommand{\cred}[1]{{\color{red}#1}}
\newcommand{\cblue}[1]{{\color{blue}#1}}

\usepackage{hyperref}
\usepackage{geometry}

\def\name{Brandon Lee}

%% The following metadata will show up in the PDF properties
\hypersetup{
  colorlinks = true,
  urlcolor = black,
  pdfauthor = {\name},
  pdfkeywords = {CS444 ``operating systems'' files filesystem I/O},
  pdftitle = {CS 444 Project 3},
  pdfsubject = {CS 444 Project 3},
  pdfpagemode = UseNone
}

\begin{document}

\begin{titlepage}
    \begin{center}
        \vspace*{3.5cm}

        \textbf{Project 3}

        \vspace{0.5cm}

        \textbf{Brandon Lee}

        \vspace{0.8cm}

        CS 444\\
        Spring 2016\\
        16 May 2016\\

        \vspace{1cm}

        \textbf{Abstract}\\

        \vspace{0.5cm}

        Block devices and encryption are essential components in developing a fundamental understanding for low level systems.  The scope of the encrypted block device problem lies an encryption problem where device module creation need to be made in order to successfully produce a device driver.  The following approach utilizes the Linux Crypto API in order to ensure valid encryption of files.  The results include a driver mounted device which utilizes the implemented block driver.


        \vfill



    \end{center}
\end{titlepage}

\newpage

\section{Design}

I will first attempt to approach the problem through trial and error starting with a basic block driver with no encryption (sbull.c).  Afterwards, I will gradually implement the encryption system through various online resources, putting it together piece by piece, tracing back when an error appears and reverting to previous commits.  Most of the cryptology will be found in a single function, which handles an IO request.  Essentially the algorithm checks block by block to see if it should encrypt or decrypt and read or write.  Additionally there will be some crypto instantiation in the init() function.

\section{Version Control Log}

\begin{tabular}{l l l}\textbf{Detail} & \textbf{Author} & \textbf{Description}\\\hline
\href{https://github.com/brandonlee503/Operating-Systems-II/commit/07ce00358a3c43cccb0873bc728f0b3dcdc5c0d7}{07ce003} & Brandon Lee & Add crypto in init() and transfer()\\\hline
\href{https://github.com/brandonlee503/Operating-Systems-II/commit/e3e4fc2d9bf79904ee2e2713e35c1f0db0d3cc80}{e3e4fc2} & Brandon Lee & First implmentation of sbd\\\hline
\href{https://github.com/brandonlee503/Operating-Systems-II/commit/432ab809f9796c4299c9d8988631461690037695}{432ab80} & Brandon Lee & Update config files\\\hline
\href{https://github.com/brandonlee503/Operating-Systems-II/commit/00b129b18f87d1ae6f1723f572e346d5a054f4e5}{00b129b} & Brandon Lee & Add writing assignment 3\\\hline
\href{https://github.com/brandonlee503/Operating-Systems-II/commit/3aff2c0cfc92a7d66fefb1f1fd3da2d4eb8e9f55}{3aff2c0} & Brandon Lee & Project 3 - Add config files\\\hline
\href{https://github.com/brandonlee503/Operating-Systems-II/commit/c34a6641307a82ff7e3a75b7ab352220d4cdde54}{c34a664} & Brandon Lee & Update naming of weekly summary 7\\\hline
\href{https://github.com/brandonlee503/Operating-Systems-II/commit/61b864eaea9f86247f5a478701335d0b9627bac1}{61b864e} & Brandon Lee & Remove Writing Assignment 2 extra files\\\hline
\href{https://github.com/brandonlee503/Operating-Systems-II/commit/d46a96bd2dcbf9fdae60d025b9b393fd583bdc79}{d46a96b} & Brandon Lee & Remove backup files\\\hline
\href{https://github.com/brandonlee503/Operating-Systems-II/commit/7923b6b7bf425b6f5cd26c8689676398f2b50636}{7923b6b} & Brandon Lee & Rename assignment 3 files and add finishing touches\\\hline
\href{https://github.com/brandonlee503/Operating-Systems-II/commit/69930131ebfccea311bd636268993ad0f9b88f2a}{6993013} & Brandon Lee & Add writing assignment 3 diagrams\\\hline
\href{https://github.com/brandonlee503/Operating-Systems-II/commit/08ff0eb46768dc2293c526f8126070a51a7fbb7b}{08ff0eb} & Brandon Lee & Add first draft of writing assignment 3\\\hline
\href{https://github.com/brandonlee503/Operating-Systems-II/commit/0c69ff67b6e5b4a4aabcae682f77d40bb382f7be}{0c69ff6} & Brandon Lee & Add Week 6 and 7 summaries\\\hline
\href{https://github.com/brandonlee503/Operating-Systems-II/commit/364a639020f9f8a745cb42dd565edc29db625c92}{364a639} & Brandon Lee & Update sleep time\\\hline
\href{https://github.com/brandonlee503/Operating-Systems-II/commit/519c330971eaf31bc32cbc543503dbb448558d20}{519c330} & Brandon Lee & Change print statements\\\hline
\href{https://github.com/brandonlee503/Operating-Systems-II/commit/cf3397d9b4dbfd76e4cd47568b99df71974537e4}{cf3397d} & Brandon Lee & Add comments\\\hline
\href{https://github.com/brandonlee503/Operating-Systems-II/commit/6957d8edeba7172d1020d1287a10dd3265df2578}{6957d8e} & Brandon Lee & Add linked list size function\\\hline
\href{https://github.com/brandonlee503/Operating-Systems-II/commit/bc03fb78725ed0ed4a8bc8b02fc00bd5eb0c2990}{bc03fb7} & Brandon Lee & Fix struct type\\\hline
\href{https://github.com/brandonlee503/Operating-Systems-II/commit/a215eba032f6fc75c62acc6a20a628ad2babbff5}{a215eba} & Brandon Lee & Clean up makefile some more\\\hline
\href{https://github.com/brandonlee503/Operating-Systems-II/commit/61cf6ee1bf911834e628d6f3d5dd743d0d7b1340}{61cf6ee} & Brandon Lee & Update makefile\\\hline
\href{https://github.com/brandonlee503/Operating-Systems-II/commit/ac7f8185769dd2d21a835d6183173e73660ff7fe}{ac7f818} & Brandon Lee & First implementation of deleter\\\hline
\href{https://github.com/brandonlee503/Operating-Systems-II/commit/d1f25cab376a6cb44f0bb4f1c6b0b3d917fc4e57}{d1f25ca} & Brandon Lee & First implementation of inserter\\\hline
\href{https://github.com/brandonlee503/Operating-Systems-II/commit/1d991e479d2e28ef279a91fff6f61228fe5aff44}{1d991e4} & Brandon Lee & First implmentation of searcher\\\hline
\href{https://github.com/brandonlee503/Operating-Systems-II/commit/ce52173ee0d65d97ece46268569a5ab749ea6de2}{ce52173} & Brandon Lee & Main function - first implmentation\\\hline
\href{https://github.com/brandonlee503/Operating-Systems-II/commit/c208911c08faab924923b23d8750be5987c71fc7}{c208911} & Brandon Lee & Update linked list struct\\\hline
\href{https://github.com/brandonlee503/Operating-Systems-II/commit/e2d51e8d76a94e1b7aff691063caabe203cd24af}{e2d51e8} & Brandon Lee & Fix comments\\\hline
\href{https://github.com/brandonlee503/Operating-Systems-II/commit/c4e97b9e39c3fdd1e75e5251806c8ff4014a2390}{c4e97b9} & Brandon Lee & Update makefile\\\hline
\href{https://github.com/brandonlee503/Operating-Systems-II/commit/4e7c7d345149160901300f36322cc74ad8a2e847}{4e7c7d3} & Brandon Lee & Concurrency 3 - Add outline and core files\\\hline
\href{https://github.com/brandonlee503/Operating-Systems-II/commit/a154077e72b345a63298daafe214af91ef36eb54}{a154077} & Brandon Lee & Add Week 5 Summary\\\hline
\href{https://github.com/brandonlee503/Operating-Systems-II/commit/0cd0eb1d541b9b592d322a34be7f9bae9e6da7ac}{0cd0eb1} & Brandon Lee & Writing Assignment 2 - First Draft Completed\\\hline
\href{https://github.com/brandonlee503/Operating-Systems-II/commit/3bcb782120ad1e9ebdd822c9f89775f330a93da9}{3bcb782} & Brandon Lee & Writing Assignment 2 - First draft\\\hline
\href{https://github.com/brandonlee503/Operating-Systems-II/commit/bec69fd13b88df6a21023c87fc89d88d8bcb1bcc}{bec69fd} & Brandon Lee & Fix section name typo\\\hline
\href{https://github.com/brandonlee503/Operating-Systems-II/commit/2ed9cdb0f476813d75f1dc61a3b63a66e6dd835e}{2ed9cdb} & Brandon Lee & Restructure patch file directories\\\hline
\end{tabular}

\newpage

\section{Work Log}

Work initially began with doing base research on what the assignment was asking, primarily using LDD3 and searching the web for additional information.  After figuring out exactly what we need to be doing, I got a basic block driver (sbull.c) started to go through much trial and error on setting up the whole system.  From building the module, to moving the module to the VM through SCP, installing the module with insmod, mounting with fdisk, making a filesystem with mkfs.ext2, and mounting with mount.  A list of commands can be found below on steps taken to get here.\\

\begin{itemize}
\item Get 3 files from patch
\item make -j4 all (Should make the .ko file)
\item Qemu - Remove of -net:none
\item scp to get the module (.ko file) into the VM
\item insmod sdb.ko
\item fdisk /dev/sbd0p1 and make a partition
\item mkfs.ext2 to make the file system (mkfs.ext2 /dev/sbd0p1)
\item mount /dev/sbd0p1 /test
\end{itemize}

Afterwards, now I that I know everything is working.  I can start with actually implementing Crypto into the device.  And through much trial and error, crypto is implemented and files can be placed into the filesystem.\\

\section{Questions}

\textit{What do you think the main point of this assignment is?}\\

I thought that the main point of this assignment was to develop a deeper understanding of Linux block devices, modules, and the Crypto API.  I also thought that the purpose of this assignment was to build an understanding on how to bring modules over into the Virtual Machine in order to establish a foundation on importing Linux modules.\\

\textit{How did you personally approach the problem? Design decisions, algorithm, etc.}\\

As I mentioned above, I first attempted to approach the problem through much trial and error starting with a basic block driver with no encryption (sbull.c).  Afterwards, I gradually implement the encryption system through various online resources, putting it together piece by piece, tracing back when an error appears and reverting to previous commits.  Most of the cryptology can be found in the transfer function, which handles an IO request.  Essentially the algorithm checks block by block to see if it should encrypt or decrypt and read or write.  As it performs this check, it either performs encryption or decryption with crypto\_cipher\_encrypt\_one() or crypto\_cipher\_decrypt\_one().  Additionally there is some crypto instantiation in the init() function.  Eventually after much trial and error, I got everything to compile and into the VM.\\

\textit{How did you ensure your solution was correct? Testing details, for instance.}\\

I wrote a script that basically does all of the above required steps to set everything up including unpatching, making the .ko file, starting the VM, bringing over the module through SCP, installing the module, partitioning the drive, cleaning it up, making a file system, and mounting it with the device.\\

\textit{What did you learn?}\\

From this assignment I learned much on Linux block devices, encrypting blocks, and writing block drivers.  I learned how to bring files into the VM through SCP.  I learned how to partition a drive, how to make a file system, and how to mount devices.  I learned how to use some vital components in the Linux Crypto API and a few other aspects as well.\\

\end{document}
