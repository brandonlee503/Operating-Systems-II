\documentclass[letterpaper,10pt,titlepage]{article}

\usepackage{graphicx}
\usepackage{amssymb}
\usepackage{amsmath}
\usepackage{amsthm}

\usepackage{alltt}
\usepackage{float}
\usepackage{color}
\usepackage{url}

\usepackage{balance}
\usepackage[TABBOTCAP, tight]{subfigure}
\usepackage{enumitem}
\usepackage{pstricks, pst-node}

\usepackage{geometry}
\geometry{textheight=8.5in, textwidth=6in}

%random comment

\newcommand{\cred}[1]{{\color{red}#1}}
\newcommand{\cblue}[1]{{\color{blue}#1}}

\usepackage{hyperref}
\usepackage{geometry}

\def\name{Brandon Lee}

%pull in the necessary preamble matter for pygments output
% \input{pygments.tex}

%% The following metadata will show up in the PDF properties
\hypersetup{
  colorlinks = true,
  urlcolor = black,
  pdfauthor = {\name},
  pdfkeywords = {cs444 ``operating systems'' files filesystem I/O},
  pdftitle = {CS 444 Weekly Summary 1},
  pdfsubject = {CS 444 Weekly Summary 1},
  pdfpagemode = UseNone
}

\begin{document}

\begin{titlepage}
    \begin{center}
        \vspace*{3.5cm}

        \textbf{Weekly Summary 1}

        \vspace{0.5cm}

        \textbf{Brandon Lee}

        \vspace{0.8cm}

        CS 444\\
        Spring 2016\\
        10 April 2016\\

        \vspace{1cm}

        \textbf{Abstract}\\

        \vspace{0.5cm}

        The objective of this document is to illustrate comprehension of the first two chapters of Robert Love's "Linux Kernel Development" (2010) through the form of succinct Rhetorical Precis.

        \vfill



    \end{center}
\end{titlepage}

\newpage

\section{Rhetorical Precis}

Robert Love's first and second chapter of "Linux Kernel Development" (2010) explains that while the Linux kernel may be powerful and unique, the actual complexity of the project is not terribly different from other software bases of similar magnitude.  Love backs up this claim by providing a history and comparison of Unix and Linux kernel versions, breaking down the installation process, and going through the common programming constructs within the kernel.  Love's purpose is to establish the notion that the Linux kernel is not as daunting as people may perceive in order to promote the learning and understanding of Linux kernel concepts.  Through the usage of technical language, Love writes to an audience of knowledgeable readers who come with a bit of computer background, but also hold an interest in learning more about lower level computer architecture.

\end{document}
