\documentclass[letterpaper,10pt,titlepage]{article}

\usepackage{graphicx}
\usepackage{amssymb}
\usepackage{amsmath}
\usepackage{amsthm}

\usepackage{alltt}
\usepackage{float}
\usepackage{color}
\usepackage{url}

\usepackage{balance}
\usepackage[TABBOTCAP, tight]{subfigure}
\usepackage{enumitem}
\usepackage{pstricks, pst-node}

\usepackage{geometry}
\geometry{textheight=8.5in, textwidth=6in}

%random comment

\newcommand{\cred}[1]{{\color{red}#1}}
\newcommand{\cblue}[1]{{\color{blue}#1}}

\usepackage{hyperref}
\usepackage{geometry}

\def\name{Brandon Lee}

%pull in the necessary preamble matter for pygments output
% \input{pygments.tex}

%% The following metadata will show up in the PDF properties
\hypersetup{
  colorlinks = true,
  urlcolor = black,
  pdfauthor = {\name},
  pdfkeywords = {cs311 ``operating systems'' files filesystem I/O},
  pdftitle = {CS 444 Weekly Summary 3},
  pdfsubject = {CS 444 Weekly Summary 3},
  pdfpagemode = UseNone
}

\begin{document}

\begin{titlepage}
    \begin{center}
        \vspace*{3.5cm}

        \textbf{Weekly Summary 3}

        \vspace{0.5cm}

        \textbf{Brandon Lee}

        \vspace{0.8cm}

        CS 444\\
        Spring 2016\\
        17 April 2016\\

        \vspace{1cm}

        \textbf{Abstract}\\

        \vspace{0.5cm}

        The objective of this document is to illustrate the comprehension of chapter fourteen in Robert Love's “Linux Kernel Development” (2010) through the form of Rhetorical Precis.

        \vfill



    \end{center}
\end{titlepage}

\newpage

\section{Rhetorical Precis}

Robert Love's fourteenth chapter of “Linux Kernel Development” (2010) explains the crucial concepts of the Block IO layer and its how its various core data structures and schedulers are essential components to a functional system.  Love supports his claims through explanations of various block layer IO fundamentals such as blocks/sectors, buffer/buffer heads, bio structures as well as multiple schedulers such as the Linus Elevator, Deadline, Anticipatory, CFQ, and Noop.  Love's purpose is to illustrate these core components in block layer and the diverse pros/cons of various schedulers in order to promote usage of the right tools for the right job.  Through the frequent usage of technical language, Love writes to an audience of knowledgeable readers who come from a background of lower level computing interest.

\end{document}
