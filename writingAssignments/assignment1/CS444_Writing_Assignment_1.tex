\documentclass[letterpaper,10pt,titlepage]{article}

% This might mess up formatting
\setlength{\parindent}{0pt}

\usepackage{graphicx}
\usepackage{amssymb}
\usepackage{amsmath}
\usepackage{amsthm}

\usepackage{alltt}
\usepackage{float}
\usepackage{color}
\usepackage{url}
\usepackage{listings}

\usepackage{balance}
\usepackage[TABBOTCAP, tight]{subfigure}
\usepackage{enumitem}
\usepackage{pstricks, pst-node}

\usepackage{geometry}
\geometry{textheight=8.5in, textwidth=6in}

\newcommand{\cred}[1]{{\color{red}#1}}
\newcommand{\cblue}[1]{{\color{blue}#1}}

\usepackage{hyperref}
\usepackage{geometry}

\lstdefinestyle{customc}{
  belowcaptionskip=1\baselineskip,
  breaklines=true,
  frame=L,
  xleftmargin=\parindent,
  language=C,
  showstringspaces=false,
  basicstyle=\footnotesize\ttfamily,
  keywordstyle=\bfseries\color{green!40!black},
  commentstyle=\itshape\color{purple!40!black},
  identifierstyle=\color{blue},
  stringstyle=\color{orange},
}

\def\name{Brandon Lee}

%pull in the necessary preamble matter for pygments output
\usepackage{fancyvrb}
\usepackage{color}
\usepackage[latin1]{inputenc}


\makeatletter
\def\PY@reset{\let\PY@it=\relax \let\PY@bf=\relax%
    \let\PY@ul=\relax \let\PY@tc=\relax%
    \let\PY@bc=\relax \let\PY@ff=\relax}
\def\PY@tok#1{\csname PY@tok@#1\endcsname}
\def\PY@toks#1+{\ifx\relax#1\empty\else%
    \PY@tok{#1}\expandafter\PY@toks\fi}
\def\PY@do#1{\PY@bc{\PY@tc{\PY@ul{%
    \PY@it{\PY@bf{\PY@ff{#1}}}}}}}
\def\PY#1#2{\PY@reset\PY@toks#1+\relax+\PY@do{#2}}

\expandafter\def\csname PY@tok@gd\endcsname{\def\PY@tc##1{\textcolor[rgb]{0.63,0.00,0.00}{##1}}}
\expandafter\def\csname PY@tok@gu\endcsname{\let\PY@bf=\textbf\def\PY@tc##1{\textcolor[rgb]{0.50,0.00,0.50}{##1}}}
\expandafter\def\csname PY@tok@gt\endcsname{\def\PY@tc##1{\textcolor[rgb]{0.00,0.25,0.82}{##1}}}
\expandafter\def\csname PY@tok@gs\endcsname{\let\PY@bf=\textbf}
\expandafter\def\csname PY@tok@gr\endcsname{\def\PY@tc##1{\textcolor[rgb]{1.00,0.00,0.00}{##1}}}
\expandafter\def\csname PY@tok@cm\endcsname{\let\PY@it=\textit\def\PY@tc##1{\textcolor[rgb]{0.25,0.50,0.50}{##1}}}
\expandafter\def\csname PY@tok@vg\endcsname{\def\PY@tc##1{\textcolor[rgb]{0.10,0.09,0.49}{##1}}}
\expandafter\def\csname PY@tok@m\endcsname{\def\PY@tc##1{\textcolor[rgb]{0.40,0.40,0.40}{##1}}}
\expandafter\def\csname PY@tok@mh\endcsname{\def\PY@tc##1{\textcolor[rgb]{0.40,0.40,0.40}{##1}}}
\expandafter\def\csname PY@tok@go\endcsname{\def\PY@tc##1{\textcolor[rgb]{0.50,0.50,0.50}{##1}}}
\expandafter\def\csname PY@tok@ge\endcsname{\let\PY@it=\textit}
\expandafter\def\csname PY@tok@vc\endcsname{\def\PY@tc##1{\textcolor[rgb]{0.10,0.09,0.49}{##1}}}
\expandafter\def\csname PY@tok@il\endcsname{\def\PY@tc##1{\textcolor[rgb]{0.40,0.40,0.40}{##1}}}
\expandafter\def\csname PY@tok@cs\endcsname{\let\PY@it=\textit\def\PY@tc##1{\textcolor[rgb]{0.25,0.50,0.50}{##1}}}
\expandafter\def\csname PY@tok@cp\endcsname{\def\PY@tc##1{\textcolor[rgb]{0.74,0.48,0.00}{##1}}}
\expandafter\def\csname PY@tok@gi\endcsname{\def\PY@tc##1{\textcolor[rgb]{0.00,0.63,0.00}{##1}}}
\expandafter\def\csname PY@tok@gh\endcsname{\let\PY@bf=\textbf\def\PY@tc##1{\textcolor[rgb]{0.00,0.00,0.50}{##1}}}
\expandafter\def\csname PY@tok@ni\endcsname{\let\PY@bf=\textbf\def\PY@tc##1{\textcolor[rgb]{0.60,0.60,0.60}{##1}}}
\expandafter\def\csname PY@tok@nl\endcsname{\def\PY@tc##1{\textcolor[rgb]{0.63,0.63,0.00}{##1}}}
\expandafter\def\csname PY@tok@nn\endcsname{\let\PY@bf=\textbf\def\PY@tc##1{\textcolor[rgb]{0.00,0.00,1.00}{##1}}}
\expandafter\def\csname PY@tok@no\endcsname{\def\PY@tc##1{\textcolor[rgb]{0.53,0.00,0.00}{##1}}}
\expandafter\def\csname PY@tok@na\endcsname{\def\PY@tc##1{\textcolor[rgb]{0.49,0.56,0.16}{##1}}}
\expandafter\def\csname PY@tok@nb\endcsname{\def\PY@tc##1{\textcolor[rgb]{0.00,0.50,0.00}{##1}}}
\expandafter\def\csname PY@tok@nc\endcsname{\let\PY@bf=\textbf\def\PY@tc##1{\textcolor[rgb]{0.00,0.00,1.00}{##1}}}
\expandafter\def\csname PY@tok@nd\endcsname{\def\PY@tc##1{\textcolor[rgb]{0.67,0.13,1.00}{##1}}}
\expandafter\def\csname PY@tok@ne\endcsname{\let\PY@bf=\textbf\def\PY@tc##1{\textcolor[rgb]{0.82,0.25,0.23}{##1}}}
\expandafter\def\csname PY@tok@nf\endcsname{\def\PY@tc##1{\textcolor[rgb]{0.00,0.00,1.00}{##1}}}
\expandafter\def\csname PY@tok@si\endcsname{\let\PY@bf=\textbf\def\PY@tc##1{\textcolor[rgb]{0.73,0.40,0.53}{##1}}}
\expandafter\def\csname PY@tok@s2\endcsname{\def\PY@tc##1{\textcolor[rgb]{0.73,0.13,0.13}{##1}}}
\expandafter\def\csname PY@tok@vi\endcsname{\def\PY@tc##1{\textcolor[rgb]{0.10,0.09,0.49}{##1}}}
\expandafter\def\csname PY@tok@nt\endcsname{\let\PY@bf=\textbf\def\PY@tc##1{\textcolor[rgb]{0.00,0.50,0.00}{##1}}}
\expandafter\def\csname PY@tok@nv\endcsname{\def\PY@tc##1{\textcolor[rgb]{0.10,0.09,0.49}{##1}}}
\expandafter\def\csname PY@tok@s1\endcsname{\def\PY@tc##1{\textcolor[rgb]{0.73,0.13,0.13}{##1}}}
\expandafter\def\csname PY@tok@sh\endcsname{\def\PY@tc##1{\textcolor[rgb]{0.73,0.13,0.13}{##1}}}
\expandafter\def\csname PY@tok@sc\endcsname{\def\PY@tc##1{\textcolor[rgb]{0.73,0.13,0.13}{##1}}}
\expandafter\def\csname PY@tok@sx\endcsname{\def\PY@tc##1{\textcolor[rgb]{0.00,0.50,0.00}{##1}}}
\expandafter\def\csname PY@tok@bp\endcsname{\def\PY@tc##1{\textcolor[rgb]{0.00,0.50,0.00}{##1}}}
\expandafter\def\csname PY@tok@c1\endcsname{\let\PY@it=\textit\def\PY@tc##1{\textcolor[rgb]{0.25,0.50,0.50}{##1}}}
\expandafter\def\csname PY@tok@kc\endcsname{\let\PY@bf=\textbf\def\PY@tc##1{\textcolor[rgb]{0.00,0.50,0.00}{##1}}}
\expandafter\def\csname PY@tok@c\endcsname{\let\PY@it=\textit\def\PY@tc##1{\textcolor[rgb]{0.25,0.50,0.50}{##1}}}
\expandafter\def\csname PY@tok@mf\endcsname{\def\PY@tc##1{\textcolor[rgb]{0.40,0.40,0.40}{##1}}}
\expandafter\def\csname PY@tok@err\endcsname{\def\PY@bc##1{\setlength{\fboxsep}{0pt}\fcolorbox[rgb]{1.00,0.00,0.00}{1,1,1}{\strut ##1}}}
\expandafter\def\csname PY@tok@kd\endcsname{\let\PY@bf=\textbf\def\PY@tc##1{\textcolor[rgb]{0.00,0.50,0.00}{##1}}}
\expandafter\def\csname PY@tok@ss\endcsname{\def\PY@tc##1{\textcolor[rgb]{0.10,0.09,0.49}{##1}}}
\expandafter\def\csname PY@tok@sr\endcsname{\def\PY@tc##1{\textcolor[rgb]{0.73,0.40,0.53}{##1}}}
\expandafter\def\csname PY@tok@mo\endcsname{\def\PY@tc##1{\textcolor[rgb]{0.40,0.40,0.40}{##1}}}
\expandafter\def\csname PY@tok@kn\endcsname{\let\PY@bf=\textbf\def\PY@tc##1{\textcolor[rgb]{0.00,0.50,0.00}{##1}}}
\expandafter\def\csname PY@tok@mi\endcsname{\def\PY@tc##1{\textcolor[rgb]{0.40,0.40,0.40}{##1}}}
\expandafter\def\csname PY@tok@gp\endcsname{\let\PY@bf=\textbf\def\PY@tc##1{\textcolor[rgb]{0.00,0.00,0.50}{##1}}}
\expandafter\def\csname PY@tok@o\endcsname{\def\PY@tc##1{\textcolor[rgb]{0.40,0.40,0.40}{##1}}}
\expandafter\def\csname PY@tok@kr\endcsname{\let\PY@bf=\textbf\def\PY@tc##1{\textcolor[rgb]{0.00,0.50,0.00}{##1}}}
\expandafter\def\csname PY@tok@s\endcsname{\def\PY@tc##1{\textcolor[rgb]{0.73,0.13,0.13}{##1}}}
\expandafter\def\csname PY@tok@kp\endcsname{\def\PY@tc##1{\textcolor[rgb]{0.00,0.50,0.00}{##1}}}
\expandafter\def\csname PY@tok@w\endcsname{\def\PY@tc##1{\textcolor[rgb]{0.73,0.73,0.73}{##1}}}
\expandafter\def\csname PY@tok@kt\endcsname{\def\PY@tc##1{\textcolor[rgb]{0.69,0.00,0.25}{##1}}}
\expandafter\def\csname PY@tok@ow\endcsname{\let\PY@bf=\textbf\def\PY@tc##1{\textcolor[rgb]{0.67,0.13,1.00}{##1}}}
\expandafter\def\csname PY@tok@sb\endcsname{\def\PY@tc##1{\textcolor[rgb]{0.73,0.13,0.13}{##1}}}
\expandafter\def\csname PY@tok@k\endcsname{\let\PY@bf=\textbf\def\PY@tc##1{\textcolor[rgb]{0.00,0.50,0.00}{##1}}}
\expandafter\def\csname PY@tok@se\endcsname{\let\PY@bf=\textbf\def\PY@tc##1{\textcolor[rgb]{0.73,0.40,0.13}{##1}}}
\expandafter\def\csname PY@tok@sd\endcsname{\let\PY@it=\textit\def\PY@tc##1{\textcolor[rgb]{0.73,0.13,0.13}{##1}}}

\def\PYZbs{\char`\\}
\def\PYZus{\char`\_}
\def\PYZob{\char`\{}
\def\PYZcb{\char`\}}
\def\PYZca{\char`\^}
\def\PYZam{\char`\&}
\def\PYZlt{\char`\<}
\def\PYZgt{\char`\>}
\def\PYZsh{\char`\#}
\def\PYZpc{\char`\%}
\def\PYZdl{\char`\$}
\def\PYZti{\char`\~}
% for compatibility with earlier versions
\def\PYZat{@}
\def\PYZlb{[}
\def\PYZrb{]}
\makeatother


%% The following metadata will show up in the PDF properties
\hypersetup{
  colorlinks = true,
  urlcolor = black,
  pdfauthor = {\name},
  pdfkeywords = {CS444 ``Operating Systems'' files filesystem I/O},
  pdftitle = {CS 444 Writing Assignment 1},
  pdfsubject = {CS 444 Writing Assignment 1},
  pdfpagemode = UseNone
}

\begin{document}

\begin{titlepage}
    \begin{center}
        \vspace*{3.5cm}

        \textbf{Writing Assignment 1}

        \vspace{0.5cm}

        \textbf{Brandon Lee}

        \vspace{0.8cm}

        CS 444\\
        Spring 2016\\
        22 April 2016\\

        \vspace{1cm}

        \textbf{Abstract}\\

        \vspace{0.5cm}

        FreeBSD, Linux, and Windows each take fundamentally different approaches to what they regard as a low level operating system.  However, within these major disparities and differences lie some similarities and parallels.  The following analysis assesses to what extent do these operating systems both bridge and dispute in the areas of processes, threading, and CPU scheduling.


        \vfill



    \end{center}
\end{titlepage}

\newpage

\tableofcontents

\newpage

\section{Processes}

In the realm of computing, the process is essentially an instance of a program that is running in execution.  Not only is a process the running program, but the open resources that it requires are also included such as necessary files, processer state, threads, and data section to name a few.  Basically, the process is the combination source code as well as all of the external resources needed for execution.


\subsection{Similarities}

To a great extent, the contents of processes in FreeBSD, Windows, and Linux all share some strong similarities with each other.  In particular, throughout these three operating systems, processes contain vital components such as their program code, private virtual address space, files, signals, and threads.  Process descriptors keep track of kernel stacks, CPU time, file descriptor tables, and machine registers.  In FreeBSD and Linux, processes are represented a structure known as a task\_struct while in Windows, they are represented by an Executive Process.  The Unix task\_struct includes accessible data pertaining to user space such as the stack, heap, and global variables.  The Windows Executive Process contains the Process Environment Block (PEB) which itself contains information including process heap, thread storage, and the executable image to be accessed from the user space.\cite{mwi1}\\

Background processes are also a major component supported by each operating system.  In the Unix-like world, background processes are known as daemons and created upon a process forking and a parent dying.  The new child process creates a new session for itself by calling setsid() and essentially frees itself from the controlling terminal.\cite{lkd4}  In the Windows realm, background processes are known as Windows Services, which are managed by a special system process known as the System Control Manager.\cite{mwi5}  Both daemons and Windows Services are typically instantiated during system startup and run longstanding until the system is powered down.  Background processes are typically utilized to run specific jobs.  For example in Linux, daemons such as syslogd and sshd, implement the core features of system logging and manage incoming logins from remote host ssh connections.\cite{syslogd} Windows background processes such as DNSCache and Spooler resolves/saves domain names to IP addresses and manage files in memory for printing respectively.\cite{mwi2}

\subsection{Differences}

One of the major differences between FreeBSD/Linux processes and Windows processes starts with the formation of a new process.  In the Linux world, a process is created with the fork() command, which essentially duplicates the current process to create a new one.  The new process is referred to as the child process while the initial is regarded as the parent.  The Linux fork() method is typically implemented with the clone() call which specifies any resources that the parent and child process need to share.  Finally the exec() call loads an executable file into the process's address space and executes it.  These multiple steps allow for a fork to perform any additional work before being loaded into the address space.  Processes in FreeBSD/Linux are schedulable and can hold states such as running, runnable, interruptible, and zombie.\cite{lkd3}\\

With Windows, a similar approach to process creation would be a call to CreateProcess.  This call essentially brings all of the above steps from Linux into one single operation to create an executable file (.exe file) and load a EPROCESS all into one step.\cite{mwi5}  Windows also includes a couple entities not found in FreeBSD/Linux known as fibers and jobs.  Fibers are threads scheduled in user space, rather than from the CPU scheduler.  A job is a group of processes combined to be managed together.  Jobs can indicate the amount of resources per thread in their processes.

\subsection{Similarities/Differences - Why?}

Process creation is the first dissimilarity between the operating systems.  The difference between the two methods of process creation between FreeBSD/Linux and Windows can be deduced from the design different patterns of the operating systems.  FreeBSD/Linux provides two separate calls for instantiating a new process and loading the executable image as the Linux design paradigm is oriented around modularity.  Each function primarily focuses on one aspect of the task as the Unix philosophy states: "Do one thing and do it well".\cite{philosophy}  Windows does not share this philosophy and keeps these process creation functions connected.  Windows does not see a benefit for the modularity as it does not expect any changes in the functionality of CreateProcess.\\

Process address space are similar in FreeBSD/Linux, and Windows as it is necessary for these operating systems to provide some sort of structure in running individual programs with the abstraction of unlimited processer usage to get the most out of those processes.  This has enabled many developers to write a massive amount of programs as these applications do not necessary need to be concerned about the lower level aspects such as managing processer resources.  This is enabling for developers to work on higher level applications.\\

User mode and kernel mode are another paradigm that is supported by FreeBSD/Linux and Windows.  This design is important for both multitasking operating systems because if one application crashes, the system as well as other processes would be protected.  The operating system would continue to function if a program within suddenly dies.  Furthermore, applications do not need to worry about the system when performing actions.  This design scheme is critical to the development of our modern operating systems today.\\

FreeBSD, Linux, and Windows all support background processes as they are critical to the underlying services that a modern operating system typically provides.  Daemons and Windows Services are the foundation for many applications and processes to function as they do.  Without background processes, multitasking operating systems would not be computing as they do today.\\

\section{Threads}

Threads are a very common concept in process programming.  They are essentially an abstraction to provide multiple sequences of execution within a single program, allowing for the sharing of process resources.  Threads enable concurrency within a program - which can lead to parallelism in multiprocessor machines.

\subsection{Similarities}

A key caveat to take notice of is the how FreeBSD, Windows, and Linux all implement some sort of user and kernel mode.  Basically a thread can utilize a system call that allows the kernel to run kernel code on the behalf of the thread itself.  Threads contain separate stacks and program counters for running in kernel mode.  When the thread switches from kernel to user mode, these data structures and counters are saved and restarted.  In Unix like systems, elements such as the heap, stack, and global variables held the task\_struct are available in user space, but other data such as task state are only available in the kernel space.\cite{lkd4}  Specifically in FreeBSD/Linux, threading is implemented through the POSIX standard pthreads.  The following is an example of creating a pthread in Linux/FreeBSD.\cite{pthread1}  Note the method parameters and calls:\\

\lstinputlisting[caption=pthreads in Unix, style=customc]{pthread1.c}

Meanwhile in Windows systems, information accessible in user mode such as elements of the Windows process address space including the process environment block (PEB) and thread environment block (TEB).  The actual PEB data structure is only available in kernel mode, but can be handled through precise system calls.  These Windows data structures include much of the data described in the previous processes section.\\

FreeBSD, Windows, and Linux all support threads that run only in kernel space.  FreeBSD and Linux kernel threads (Kernel threads) are instantiated when the machine is booted and do not have a separate process context or a task\_struct data structure.  Kernel threads are used to access information that is typically not open to user mode threads.  These tasks include low level jobs such as ksoftirqd to handle interrupts.  In the Windows world, these threads are known as system threads.  System threads are created with PsCreateSystemThread and do not have a TEB structure, similar to how kernel threads do not have a task\_struct structure.\cite{lkd4}  Both kernel threads and system threads do not have user process address space.\\

The following is a respective example of threading with the Windows PSCreateSystemThread method. \cite{PsCreateSystemThread} Take note of the similarities and differences between the two different paradigms through the method syntax and parameters.

\lstinputlisting[caption=PSCreateSystemThread in Windows, style=customc]{PsCreateSystemThread.c}

\subsection{Differences}

A major difference between FreeBSD, Windows, and Linux is found within application of threads.  In the FreeBSD/Linux world, threads are practically the same to typical processes, but just share resources with other process/threads.  However, in Windows processes must contain at least one thread.  It is these threads that get processed by the scheduler.  As Windows tends to group things into various data structures, the abstraction for this is known as the Executive Thread Block (ETHREAD).  To summarize, FreeBSD/Linux processes and threads are essentially the same other than the fact that threads share resources.  In Windows, processes are required to contain at least one thread in order to even run.  They are designed to be units of execution versus the overarching process, which is intended to manage resources.\cite{mwi5}  In Windows, only threads are able to be scheduled.  These thread states in Windows are similar to process states held in Linux with conditions such as running, ready, standby, waiting, terminated, initialized, and transition.\cite{lkd4}\\

\subsection{Similarities/Differences - Why?}

Threads are fundamentally different between FreeBSD, Linux, and Windows - for respectable reasons.  In FreeBSD/Linux, threads are focused on offering a method in sharing resources.  In Windows, threads are designed to perform quickly and efficiently.  Therefore in the Linux world, processes are almost essentially identical while in Windows, there exists a single data structure (ETHREAD) to represent threads.  While Unix treads have different address spaces, they share the same stack, heap, and global variables.  Windows threads are more schedulable and executable - ETHREADS do not hold resources, but include a pointer to the EPROCESS that created it.\cite{mwi5}  As a result, ETHREADS are quick and efficient.\\

In Windows, only threads are able to hold schedulable states such as the Linux process states of runnable, sleeping, uninterruptable, traced, and stopped.  These states in Windows vocabulary are listed in the above section.  Processes in Windows are used as essentially utilized as abstract groupings of threads and resources.\\

One similarity that FreeBSD/Linux and Windows do share however are their implementations of system and kernel threads.  These threads allow for jobs to be executed exclusively in kernel mode.\\

\section{CPU Scheduling}

At the heart of every multitasking operating system lies the scheduler.  CPU scheduling essentially plans out which, where, when, and how long does the processer allocate resources to between specific processes.  Essentially the goal of the scheduler is to always have the processer being utilized to maximum potential.

\subsection{Similarities}

In Linux world, the default scheduler is the Completely Fair Scheduler (CFS).  This scheduler replaces the old O(1) scheduler, but in the process also utilizes older concepts such as sleeper fairness - which compares sleeping tasks versus those on the queue to run.\\

In FreeBSD, the default scheduler is the ULE scheduler.  This paradigm divides the scheduler into a high and a low level scheduler.  The high level scheduler's function is to order threads by priority for the CPU while the low level scheduler fetches the greatest priority thread in the queue and sends it to be run. The priority style scheduling found in FreeBSD can be observed in Windows as well.\\

In the Windows realm, similarities are observable as Windows implements a type of priority system to rank various threads.  Factors that contribute to a thread's priority and affinity are assessed within a processer group, which is set using the scheduling API.  Priority is viewed by Windows through two different variables - the initial priority class for the process from original creation and individual thread priority.  Priority can also be increased and decreased depending on several possible variables.  Thread priority is not affected by its parent process in the sense that the scheduler views all threads as equal and initially attempts to give each thread equal time on the processor.\\

\subsection{Differences}

The main differences between FreeBSD, Linux, and Windows schedulers can be found in multiple areas.  Firstly, Linux uses a unified process and thread scheduler - which is CFS by default.  FreeBSD uses the ULE scheduler as default.  Windows utilizes process scheduling with a multilevel feedback queue.  The thread scheduler here recognizes high priority threads and the process scheduler wisely uses the feedback queue.  Another difference between Linux and Windows is are fibers, which are basically scheduled treads in user space rather than the kernel scheduling algorithm.\\

\subsection{Similarities/Differences - Why?}

The main similarities between FreeBSD and Windows is observable through the similarities of their priority based scheduling.  One can observe that the reasoning behind having priority based scheduling can be because it has proven in past systems to be an effective method of mostly fair scheduling.\\

However, there do still exist differences between the two scheduling approaches.  As described before, in Windows - only threads have schedulable states as the scheduler is a bit more complex in the windows world.  Windows processes are not designed to be units of execution.  The reasoning for this is because threads are more lightweight in the operating system and therefore many more can exist at a single time.  Having a greater capacity for more threads results in a stronger environment for scheduling with threads rather than processes.\\

\newpage
\bibliographystyle{plain}
\bibliography{CS444_Writing_Assignment_1}
\end{document}
